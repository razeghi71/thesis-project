
\فصل{نتیجه‌گیری}

در این بخش به جمع‌بندی مطالب گفته شده و مسائلی که در ادامه امکان کار بر روی آنها وجود دارد می‌پردازیم.

\قسمت{جمع‌بندی}

\شروع{شمارش}
\فقره 
الگوریتم با پیچیدگی زمانی بهینه‌ی 
$O(n\log n)$
برای مسئله یافتن ۱-مرکز حالت گسسته در فضای بیشین با نقاط پرت وزن‌دار
\فقره
الگوریتم با پیچیدگی زمانی
$O(n^3\log n)$
برای مسئله یافتن ۱-مرکز در فضای اقلیدسی با تابع کاهش هزینه
\فقره 
الگوریتم با پیچیدگی زمانی
$O(n^2\log n)$
برای مسئله یافتن ۱-مرکز حالت گسسته در هر فضا با تابع کاهش هزینه
\پایان{شمارش}




\قسمت{کارهای آتی}
\شروع{شمارش}
\item 
بهبود دادن الگوریتم‌های گفته شده

همان‌طور که در بخش‌های گذشته دیدیم الگوریتم‌های ارائه شده در بخش فضای بیشین به صورت بهینه مطرح شده‌اند و در نتیجه امکان بهبود آنها وجود ندارد ولی برای بخش فضای اقلیدسی الگوریتم ارائه شده از پیچیدگی زمانی 
$O(n^3\log n)$
است و همان‌طور که در الگوریتم مربوطه دیده شد به طور کلی همه حالت‌هایی که دایره بهینه می‌تواند داشته باشد را در نظر می‌گیریم و جواب را برای‌ آنها محاسبه می‌کنیم. حدسی که برای این قسمت داریم این است که در صورتی که بتوانیم به نحوی جفت نقاطی را که به هیچ عنوان امکان تشکیل دایره بهینه ندارند را پیدا کنیم می‌توانیم به الگوریتمی با پیچیدگی زمانی کمتر دست پیدا کنیم.
\item بررسی مسئله برای حالت ۲-مرکز

در بخش‌ فضای بیشین الگوریتم ارائه شده تنها برای مسئله یک مرکز کاربرد دارد در حالی که ممکن است با بررسی چند حالت بیشتر و یا دسته‌بندی نقاط به دو دسته مجزا بتوانیم الگوریتم را برای حالت ۲-مرکز تعمیم دهیم.

هم‌چنین برای مسئله فضای بیشین نیز الگوریتم تنها برای پیدا کردن یک دایره بهینه جواب را پیدا خواهد کرد. و جواب برای حالت ۲-مرکز می‌تواند کاربرد‌های بسیار بیشتری در علوم شبکه‌های کامپیوتری داشته باشد.
\item حل مسئله برای توابع ساده‌تر به طور مثال زمانی که وزن همه نقاط با هم برابر باشد

همان‌طور که دیدیم در بخش فضای اقلیدسی که تابع هزینه به ازای هر مجموعه‌ای از نقاط محاسبه می‌شد هزینه و پیچیدگی زمانی بسیار زیادی را بر الگوریتم متحمل می‌شود. حال ممکن است در صورتی که وزن همه نقاط را با یکدیگر برابر بگیریم (به عبارت دیگر فرض کنیم هزینه از دست دادن همه مشتری‌ها با یکدیگر برابر است) آنگاه ممکن است الگوریتم ساده‌تر و با پیچیدگی زمانی کمتری قابل محاسبه باشد.
\پایان{شمارش}