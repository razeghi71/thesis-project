\فصل{بیان مشکلات نرم‌افزارهای مدیریت رمز عبور}

در این فصل به طرح مسئله و مشکلاتی که نرم‌‌افزارهای مدیریت گذرواژه‌ی دارند می‌پردازیم.

\قسمت{مشکل وصل شدن به \لر{X Server}}

در سیستم‌عامل گنو/لینوکس وظیفه‌ی مدیریت دستگاه‌های ورودی به خصوص صفحه‌کلید و موس به عهده‌ی X است. با توجه به اینکه X مبتنی بر معماری مشتری-کارگزار\پانویس{\لر{Client-Server}} است، یک نرم‌افزار با وصل شدن به آن می‌تواند به راحتی کلید‌های فشرده شده بر روی کیبورد را گوش کند و یا دستور فشرده شدن یک کلید را بدهد. 

یک سناریوی متصور برای یک Keylogger این است که خود را به عنوان میانبر یک کلید کیبورد(مانند k) در نرم‌افزار مدیریت میزکار\پانویس{\لر{Desktop Manager}} ثبت‌نام کند. از آن پس هر موقع کاربر کلید k را بزند، Keylogger اجرا می‌شود و با وصل شدن به X تمامی دستورات را می‌خواند.

\قسمت{مشکل اجرای برنامه‌ی مشابه} 
\برچسب{فصل۱:مشابه}

نرم‌افزارهای مدیریت گذرواژه برای رمزکردن گذرواژه‌ها از یک گذرواژه‌ی اصلی استفاده میکنند که معمولن در ابتدای اجراشان آن را از کاربر می‌پرسند. یک نرم‌افزار می‌تواند با جعل کردن ظاهر برنامه‌ی مدیریت گذرواژه کاربر را فریب دهد و این رمز را از وی بگیرد سپس آن را به نرم‌افزار اصلی بدهد و اجرا کند. به این صورت کاربر حتی متوجه دزدیده شدن رمزش نیز نمی‌شود.

\قسمت{مشکل خواندن از تصویر صفحه}

یکی از راه‌حل‌های نرم‌افزارهای مدیریت گذرواژه برای مشکل \ارجاع{فصل۱:مشابه} این است که هنگام نصب برنامه از کاربر می‌خواهند تا علاوه بر گذرواژه‌ی اصلی، کلمه‌ی دیگری برای احراز اصالت برنامه نیز وارد کنند. از آن به بعد برنامه همیشه هنگام دریافت گذرواژه‌ی اصلی این کلمه را نیز به کاربر نشان می‌دهد، تا کاربر مطمئن شود که این برنامه همان برنامه‌ی اصلی است.

مشکل این روش این است که یک نرم‌افزار می‌تواند با وصل شدن به \لر{X Server} تصویر صفحه‌ی فعلی را دریافت کند (این همان کاری است که نرم افزارهای \لر{ScreenShot} میکنند). کافی است یک بار که نرم‌افزار اصلی مدیریت گذرواژه اجرا شد تصویری از صفحه گرفته شود و سپس با یک پردازش تصویر ساده، کلمه‌ی انتخاب شده توسط کاربر پیدا می‌شود.

\قسمت{مشکل رفتن گذرواژه روی دیسک سخت}

قابلیت Swap در سیستم‌عامل‌ها این اجازه را به کاربر می‌دهد تا از بخشی از دیسک سخت خود به عنوان حافظه‌ی تصادفی استفاده کند. با توجه به اینکه نرم‌افزارهای مدیریت رمز عبور اطلاعات حساسی مانند گذرواژه ها را در خود نگاه می‌دارند. اگر این اطلاعات بر روی دیسک برود و بالفرض کامپیوتر خاموش شود. بعداً این گذرواژه از روی دیسک به راحتی قابل بازیابی است.

\قسمت{مشکل Dumpable بودن نرم‌افزار}

در سیستم عامل گنو/لینوکس یکی از قابلیت‌هایی که برای رفع مشکل نرم‌افزارها تعبیه شده. قابلیت گرفتن محتوای حافظه‌ی\پانویس{Core Dump} یک برنامه است. در صورتی که این قابلیت فعال باشد. یک برنامه‌ی debugger میتواند به نرم‌افزار وصل شود و محتوای حافظه‌ی آن را که شامل گذرواژه‌ها و به خصوص گذرواژه‌ی اصلی است را بخواند.

