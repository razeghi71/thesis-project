\فصل{بیان مشکلات نرم‌افزارهای مدیریت رمز عبور}

در این فصل به طرح مسئله و مشکلاتی که نرم‌‌افزارهای مدیریت گذرواژه‌ی دارند می‌پردازیم.

\قسمت{مشکل وصل شدن کلیددزد به \لر{X Server}}

در سیستم‌عامل گنو/لینوکس وظیفه‌ی مدیریت دستگاه‌های ورودی به خصوص صفحه‌کلید و موس به عهده‌ی \لر{X} است. با توجه به اینکه \لر{X} مبتنی بر معماری مشتری-کارگزار\پانویس{\لر{Client-Server}} است، یک نرم‌افزار با وصل شدن به آن می‌تواند به راحتی کلید‌های فشرده شده بر روی صفحه‌کلید را گوش کند و یا دستور فشرده شدن یک کلید را بدهد. 

یک سناریوی متصور برای یک کلیددزد این است که خود را به عنوان میانبر\پانویس{Shortcut} یک کلید صفحه‌کلید(مانند \لر{k}) در نرم‌افزار مدیریت میزکار\پانویس{Desktop Manager} ثبت‌نام کند. از آن پس هر موقع کاربر کلید \لر{k} را بزند، کلیددزد اجرا می‌شود و با وصل شدن به \لر{X} تمامی دستورات را می‌خواند.

\قسمت{مشکل اجرای برنامه‌ی مشابه} 

نرم‌افزارهای مدیریت گذرواژه برای رمزکردن گذرواژه‌ها از یک گذرواژه‌ی اصلی استفاده میکنند که معمولاً در ابتدای اجراشان آن را از کاربر می‌پرسند. یک نرم‌افزار می‌تواند با جعل کردن ظاهر برنامه‌ی مدیریت گذرواژه کاربر را فریب دهد و این رمز را از وی بگیرد سپس آن را به نرم‌افزار اصلی بدهد و اجرا کند. به این صورت کاربر حتی متوجه دزدیده شدن رمزش نیز نمی‌شود.

\قسمت{مشکل رفتن گذرواژه روی دیسک سخت}

قابلیت Swap در سیستم‌عامل‌ها این اجازه را به کاربر می‌دهد تا از بخشی از دیسک سخت خود به عنوان حافظه‌ی تصادفی استفاده کند. با توجه به اینکه نرم‌افزارهای مدیریت رمز عبور اطلاعات حساسی مانند گذرواژه ها را در خود نگاه می‌دارند. اگر این اطلاعات بر روی دیسک برود و بالفرض کامپیوتر خاموش شود. بعداً این گذرواژه از روی دیسک به راحتی قابل بازیابی است.

\قسمت{مشکل \لر{Dumpable} بودن نرم‌افزار}

در سیستم عامل گنو/لینوکس یکی از قابلیت‌هایی که برای رفع مشکل نرم‌افزارها تعبیه شده، قابلیت گرفتن محتوای حافظه‌ی\پانویس{Core Dump} یک برنامه است. در صورتی که این قابلیت فعال باشد. یک برنامه‌ی \لر{Debugger} میتواند به نرم‌افزار وصل شود و محتوای حافظه‌ی آن را که شامل گذرواژه‌ها و به خصوص گذرواژه‌ی اصلی است را بخواند.

\قسمت{مشکل احراز هویت برنامه‌ها از طریق واسط برنامه‌نویسی}

برنامه‌هایی که از طریق واسط کاربری به مدیر گذرواژه متصل می‌شوند باید احراز هویت شوند تا برنامه‌ای نتواند گذرواژه‌ها برنامه‌ی دیگر را بخواند. از طرفی احراز هویت با شناسه‌ی کاربری و گذرواژه برای مشتری‌ها کاری بی‌فایده است. زیرا نرم‌افزار مدیریت گذرواژه آمده است تا گذرواژه‌ها را ذخیره مدیریت کند. ، نه اینکه خودش به ازای هر برنامه شناسه‌ی کاربری و گذرواژه‌ی جدید بسازد که خود مشتری از آن‌ها مراقبت کند. 


