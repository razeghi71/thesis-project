\فصل{تابع فضای اقلیدسی}
در این بخش به بررسی مسئله مکان‌یابی تسهیلات تحت تابع اندازه‌ی اقلیدسی می‌پردازیم.

در بخش کارهای پیشین درباره فضای اقلیدسی و مسائلی که درباره این بخش حل شده است به طور مفصل توضیح داده شد. 

نتایجی که در این بخش به دست آمده است مربوط به مسئله پیدا کردن جواب بهینه با تابع هزینه است. 

\قسمت{ انعکاس }
\شروع{تعریف}
فرض کنید نقطه‌ای در فضا به نام $O$ داده شده باشد. تبدیل انعکاس \پانویس{reflection} نسبت به $O$ با شعاع انعکاس یک به این صورت تعریف می‌شود:
\begin{enumerate}
\item
$
\overline{OP}=\frac{1}{\overline{OP'}}
$ 
\item
نقاط $ P $ و $ P'$ در یک سمت $O$ قرار دارند.
\item
نقطه $P'$ روی پاره‌خط $OP$ قرار دارد.
\end{enumerate}
\پایان{تعریف}
به طور مثال همان‌ گونه که در شکل~\رجوع{fig:reflection} می‌بینید انعکاس یک دایره که از مرکز انعکاس می‌گذرد یک خط خواهد شد.

\شروع{شکل}[h]
\تنظیم‌ازوسط
\درج‌تصویر[width=10cm]{reflection}
\شرح{مثالی از مسئله‌ی انعکاس}
\برچسب{fig:reflection}
\پایان{شکل}

در ادامه به بررسی تعدادی از قضیه‌های مربوط به انعکاس می‌پردازیم.
\شروع{قضیه}
منعکس خطی که از مرکز انعکاس نگذرد ، دایره ای است که از مرکز انعکاس می گذرد و بر عکس.
\پایان{قضیه}

\شروع{قضیه}
منعکس دایره ای از مرکز انعکاس نگذرد. یک دایره است.
\پایان{قضیه}

\شروع{قضیه}
\برچسب{r2d}
اگر شعاع دایره‌ای که از مرکز انعکاس می‌گذرد را $r$ و فاصله مرکز انعکاس تا دایره منعکس شده (که برابر خط است) را $d$ بنامیم معادله زیر برقرار است.

$$
r=\frac{1}{2d}
$$

\پایان{قضیه}
\قسمت{تعریف مسئله}

در این بخش مسئله‌ای که می‌خواهیم به ارائه الگوریتمی برای آن بپردازیم به شرح زیر است:\\

\textbf{
تعدادی نقطه وزن‌دار در صفحه داده شده‌اند. هدف یافتن دایره‌ای است که مجموع شعاع آن و جمع وزن‌های نقاط پوشیده نشده کمینه باشد.
}
برای ارائه الگوریتم در این بخش از تبدیل انعکاس برای ساده‌تر کردن فضای مسئله استفاده می‌کنیم. این تبدیل با استفاده از قضیه زیر که اجتماعی از قضایای گفته شده در قسمت‌های قبل است انجام می‌شود.

\شروع{قضیه}
فرض کنید که دایره بهینه را به دست آورده‌ایم. این دایره توسط حداقل دو نقطه از نقاط داده شده محدود شده است. این دو نقطه را در نظر بگیرید. یکی از آنها را به عنوان مرکز انعکاس در نظر می‌گیریم. دایره بهینه بعد از تبدیل انعکاس تبدیل به خطی خواهد شد که از نقطه دیگر می‌گذرد و شعاع دایره با عکس فاصله مرکز انعکاس تا خط مذکور متناسب است.
\پایان{قضیه}
\شروع{اثبات}
دایره بهینه حداقل توسط دو نقطه محدود شده است. اگر این طور نباشد و تنها توسط یک نقطه یا کمتر محدود شده باشد به سادگی می‌توان دید که با کوچکتر کردن شعاع به اندازه $\eps$ اندازه دایره کوچک شده ولی هم‌چنان تمام نقاط قبلی را می‌پوشاند که مغایر فرض بهینگی است.
همان‌طور که در قبل گفته شد تبدیل دایره‌ای که از مرکز انعکاس می‌گذرد برابر با یک خط خواهد بود. از آنجایی که نقطه دیگر که روی دایره است به همراه دایره تبدیل داده خواهد شد در نتیجه قسمت دوم قضیه نیز اثبات می‌شود.
قسمت سوم قضیه نیز به طور مستقیم از قضیه \رجوع{r2d} نتیجه می‌شود. در شکل \رجوع{fig:masale} نمونه‌ای از این تبدیل نمایش داده شده است.
\پایان{اثبات}

\شروع{شکل}[h]
\تنظیم‌ازوسط
\درج‌تصویر[width=10cm]{masale}
\شرح{مثالی دیگر از مسئله‌ی انعکاس}
\برچسب{fig:masale}
\پایان{شکل}

\قسمت{الگوریتم ارائه شده}
در این قسمت به توضیح الگوریتم ارائه شده می‌پردازیم. برای شروع به ازای هر دو نقطه‌ای از $n$ نقطه داده شده باید الگوریتم زیر را تکرار کنیم. دلیل آن نیز است که می‌خواهیم همه حالتهایی که دایره بهینه می‌تواند داشته باشد را در نظر بگیریم. به عبارت دیگر می‌دانیم که دایره بهینه توسط حداقل دو نقطه محدود شده است. بنابراین برای اینکه همه دایره‌های بهینه را در نظر بگیریم کافی است به ازای هر دو نقطه فرض کنیم که دایره بهینه از آن دو گذشته است و مسئله را برای این دایره مفروض حل کنیم.

سپس همه نقاط را به صورت زاویه‌ای بر حسب زاویه آنها نسبت به مرکز انعکاس مرتب‌سازی می‌کنیم. روش محاسبه این زوایا نیز به این شکل است که زاویه پدید آمده بین نقطه مذکور با دو نقطه روی دایره را حساب کرده و نقاط را بر حسب این مقدار به دست آمده‌شان ترتیب می‌دهیم. در شکل \رجوع{fig:sort} نمونه‌ای از این شیوه مرتب‌سازی آمده است.

سپس به ترتیب روی ترتیب به دست آمده نقاط حرکت می‌کنیم و خط واصل بین نقطه روی دایره (که مرکز انعکاس نیست) و نقطه مذکور را به عنوان خط بهینه (و یا دایره بهینه در فضای اصلی مسئله) در نظر می‌گیریم. در نهایت نیز می‌دانیم که با هر تغییر زاویه (یا حرکت بر روی ترتیب نقاط به دست آمده) دقیقا یک نقطه از مجموعه نقاط ما کم و یا اضافه می‌شود. در صورتی که نقطه ای که از روی آن عبور می‌کنیم سمت چپ خط به دست آمده باشد یعنی داخل دایره افتاده است و در صورتی که سمت راست آن باشد یعنی خارج دایره افتاده است.\\


\begin{algorithm}
\caption{پیدا کردن شعاع بهینه}
\begin{algorithmic}[1]
\For{ هر دو نقطه از نقاط داده شده}
\State{
یکی از دو نقطه را $p$ می‌نامیم و به عنوان مرکز انعکاس در نظر می‌گیریم و نقطه دیگر را $q$ می‌نامیم.
}
\For{هر نقطه $r$}
\State{
زاویه $rqp$ را محاسبه می‌کنیم و نقاط را بر حسب زاویه به دست ‌آمده‌شان مرتب می‌کنیم.
}
\EndFor
%\State{
%از خط $pq$ شروع می‌کنیم.
%}
\For{
هر نقطه به ترتیب زاویه
}
\State{
$s$ را برابر نقطه ابتدای صف قرار می‌دهیم.
}
\State{
خط $qs$ را در نظر می‌گیریم. 
}
\State{
 دقیقا یک نقطه به مجموعه نقاطی که با این خط پوشیده می‌شوند اضافه یا کم شده است.
}
\State{
تابع هزینه این حالت را محاسبه می‌کنیم.
}
\EndFor
\EndFor
\State{
بین حالت‌های محاسبه شده کمینه را به عنوان جواب برمی‌گردانیم.
}
\end{algorithmic}
\end{algorithm}

\شروع{شکل}[h]
\تنظیم‌ازوسط
\درج‌تصویر[width=7cm]{sort}
\شرح{مرتب‌سازی زاویه‌ای}
\برچسب{fig:sort}
\پایان{شکل}

\قسمت{پیچیدگی محاسباتی}
\شروع{قضیه}
مسئله مطرح شده با استفاده از تبدیل انعکاس در زمان
$O(n^3\log n)$
قابل حل است.
\پایان{قضیه}
\شروع{اثبات}
خطوط ۲ تا ۹ به ازای هر دو نقطه اجرا خواهند شد. در نتیجه در زمان 
$O(n^2)$
بار تکرار خواهند داشت.

 خطوط ۳ و ۴ زاویه مورد نظر را به ازای همه نقاط به دست‌ آورده و سپس مرتب‌سازی می‌کنند که در نتیجه در زمان 
$O(n\log n)$
انجام خواهد شد. 

خطوط ۵ الی ۹ نیز به ازای هر نقطه یک بار انجام خواهد شد که از 
$O(n)$
خواهد بود.
در نتیجه پیچیدگی نهایی این الگوریتم برابر است با 
$O(n^2\times(n\log n+n))=O(n^3\log n)$.
\پایان{اثبات}
