\فصل{بررسی چندین نمونه نرم‌افزار مدیریت گذرواژه}

در این فصل چند نرم‌افزار مدیریت گذرواژه در لینوکس را به اخصتار معرفی میکنیم. 

\قسمت{نرم‌افزار \لر{KeePassX}}

این نرم‌افزار یک نسخه‌ی غیررسمی از روی نرم افزار ویندوزی \لر{KeePass} می‌باشد و قابلیت ساختن چندین پایگاه‌داده با گذرواژه‌های اصلی متفاوت داراست. \لر{KeePassX} از \لر{AES} و \لر{Twofish} برای رمز‌نگاری داده‌ها استفاده میکند. 

با کلیک راست بر روی هر موجودیت ذخیره شده می‌توانید شناسه‌ی کاربری و گذرواژه را در \لر{ClipBoard} کپی کرده تا در مرورگر استفاده کنید. واضحاً هر نرم‌افزاری می‌تواند محتوای \لر{ClipBoard} را بخواند و این امر یک تهدید امنیتی محسوب می‌شود.

\شروع{شکل}[h]
\تنظیم‌ازوسط
\درج‌تصویر{keepassx}
\برچسب{fig:keepassx}
\شرح{تصویری از نرم‌افزار \لر{keepassx}}
\پایان{شکل}

\قسمت{نرم افزار\لر{Gpassword Manager}}


