\فصل{معرفی چندین نمونه نرم‌افزار مدیریت گذرواژه}

در این فصل چند نرم‌افزار مدیریت گذرواژه در لینوکس را به اخصتار معرفی میکنیم. 

\قسمت{نرم‌افزار \لر{KeePassX}}

این نرم‌افزار یک نسخه‌ی غیررسمی از روی نرم افزار ویندوزی \لر{KeePass} می‌باشد و قابلیت ساختن چندین پایگاه‌داده با گذرواژه‌های اصلی متفاوت داراست. \لر{KeePassX} از \لر{AES} و \لر{Twofish} برای رمز‌نگاری داده‌ها استفاده میکند. 

با کلیک راست بر روی هر موجودیت ذخیره شده می‌توانید شناسه‌ی کاربری و گذرواژه را در \لر{ClipBoard} کپی کرده تا در مرورگر استفاده کنید. واضحاً هر نرم‌افزاری می‌تواند محتوای \لر{ClipBoard} را بخواند و این امر یک تهدید امنیتی محسوب می‌شود.

\شروع{شکل}[h]
\تنظیم‌ازوسط
\درج‌تصویر{keepassx}
\برچسب{fig:keepassx}
\شرح{تصویری از نرم‌افزار \لر{keepassx}}
\پایان{شکل}

\قسمت{نرم‌افزار \لر{Encryptr}}

\لر{Encryptr} مدیر گذرواژه‌ی آزاد روی ابر است. این نرم‌افزار با استفاده از چارچوب\پانویس{framework} دانایی صفرِ \لر{Crypton} نوشته شده است و در حال حاضر برای سیستم‌عامل‌های گنو/لینوکس، ویندوز، مک، آندروید و \لر{iOS} نرم‌افزار دارد. 

\لر{Encryptr} عملیات رمزگذاری و رمزگشایی داده‌ها را به صورت محلی\پانویس{local} انجام می‌دهد و تنها چیزی که روی ابر ذخیره می‌شود، داده‌های رمز‌گذاری شده است.


\شروع{شکل}[h]
\تنظیم‌ازوسط
\درج‌تصویر{encryptr}
\برچسب{fig:encryptr}
\شرح{تصویری از نرم‌افزار \لر{encryptr}}
\پایان{شکل}






