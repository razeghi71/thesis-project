\فصل{کارهای پیشین}
\برچسب{chapter:previous}

مسئله‌ی پیدا کردن کوچک‌ترین دایره‌ی پوشاننده‌ی مجموعه‌ای از نقاط در صفحه، اولین بار توسط سیلوستر \مرجع{sylvester1857question} در سال ۱۸۷۵ مطرح شد. وی الگوریتمی با پیچیدگی زمانی $O(n^4)$ برای این مسئله ارائه داد که بر آزمایش تک‌تک دایره‌های کاندید مبتنی بود. در سال ۱۹۶۷ توسط بیس \مرجع{bass1967finding} تلاش‌هایی جهت به‌بود عملی این الگوریتم انجام شد که البته مرتبه‌ی زمانی آن را به لحاظ نظری به‌بودی نداد. ایده‌ی کلی این روش‌ها هرس کردن نقاط و دور ریختن بخشی آن‌ها بود که تاثیری در جواب نهایی نداشتند. به عنوان مثال، بیس مشاهده کرد که تنها نقاطی که در جواب این مسئله تاثیر دارند، نقاط مفرط \پانویس{extereme points} ورودی هستند. در سال ۱۹۷۲ الزینگا \مرجع{elzinga1972geometrical} الگوریتمی از مرتبه‌ی زمانی $O(n^3)$ ارائه می‌دهد.

در سال ۱۹۷۵، الگوریتمی توسط شموس \مرجع{shamos1975closest} ارائه شد که مرتبه‌ی زمانی حل این مسئله را به $O(n \log n)$ کاهش داد که پیش‌رفت چشم‌گیری نسبت به الگوریتم قبلی داشت. وی در همین مقاله پیش‌بینی کرد که این مرتبه‌ی زمانی برای حل این مسئله بهینه است. بعدها این حدس مردود شد. الگوریتمی که شموس در این مقاله ارائه داد ابتدا نمودار ورونوی \پانویس{Voronoi diagram} دورترین نقاط\پانویس{farthest point} \مرجع{fortune1987sweepline} نقاط را به دست می‌آورد، سپس با توجه به این که مرکز دایره‌ی بهینه، باید روی یکی از رئوس نمودار یا در محل تقاطع پاره‌خط واصل دو نقطه و یال بین ناحیه‌ی مربوط به آن دو نقطه باشد، تعداد کاندیدهای محل قرارگیری مرکز را به $O(n)$ کاهش می‌دهد. شعاع دایره‌ی محیطی نقاط به مرکز هر یکی از این کاندیدها در زمان ثابت انجام‌پذیر است. پس این الگوریتم با بررسی تک‌تک کاندیدها، می‌تواند جواب بهینه را در $O(n \log n)$ پیدا کند.

در سال ۱۹۸۳، مگیدو \مرجع{megiddo1982linear} الگوریتمی خطی برای حل این مسئله ارائه کرد که علاوه بر مردود کردن حدس شموس مبنی بر کران پایین $\Omega(n \log n)$ برای مسئله‌ی $1$-مرکز، الگوریتمی بهینه برای حل این مسئله ارائه داد. الگوریتم مگیدو مبنی بر هرس کردن مکرر نقاط است و در هر مرحله $\frac{1}{16}$ نقاط ورودی را دور می‌ریزد، طوری که جواب نهایی تغییری نکند. نهایتاً جواب مسئله را برای تعداد ثابت نقاط باقی‌مانده حل می‌کند.

در سال ۱۹۸۴، درزنر \مرجع{drezner1984planar} مسئله‌ی $2$-مرکز را در صفحه‌ی دوبعدی مورد بررسی قرار می‌دهد و موفق می‌شود الگوریتمی با مرتبه‌ی زمانی $O(n^3)$ برای این مسئله ارائه دهد.

در سال ۱۹۸۶، لی مسئله‌ی $1$-مرکز گسسته را بررسی کرده و الگوریتمی از مرتبه‌ی زمانی $O(n \log n)$ ارائه می‌دهد. وی همچنین نشان می‌دهد که این الگوریتم تحت مدل محاسباتی درخت محاسبه‌ی بِن‌اُر \مرجع{ben1983lower}، بهینه است. برای حل مسئله، او از روش ساختن نمودار ورونوی نقاط، که قبلاً توسط شموس مطرح شده بود، استفاده می‌کند. برای اثبات بهینگی این روش نیز از کاهش مسئله‌ی برابری مجموعه استفاده کرده و با استفاده از لم بن‌اُور \مرجع{ben1983lower} نتیجه می‌گیرد که این مسئله در زمان به‌تر از $O(n \log n)$ قابل حل نیست.

شازل \مرجع{chazelle1986circle} در سال ۱۹۸۶، مسئله‌ی مکان‌یابی دایره با شعاع واحد را بررسی می‌کند. در این مسئله $n$ نقطه در صفحه به عنوان ورودی داده شده‌اند و هدف قرار دادن یک دایره‌ی واحد در صفحه است، طوری که تعداد نقاط پوشش داده شده بیشینه گردد. شازل الگوریتم مبتکرانه‌ای با مرتبه‌ی زمانی $O(n^2)$ برای این مسئله ارائه داده است. این الگوریتم برای تصمیم‌گیری بزرگی یا کوچکی یک $r$ نسبت به $r^*$ مکرراً استفاده شده است و کاربرد آن در سایر مقالات این حوزه شایان توجه است. شازل نشان می‌دهد که همین الگوریتم برای نقاط وزن‌دار نیست قابل استفاده است.

در سال ۱۹۸۷، درنزر \مرجع{drezner1987rectangular} الگوریتمی خطی برای حل مسئله‌ی $2$-مرکز در فضای $l_\infty$ ارائه می‌دهد. این الگوریتم نقاط را به‌طور حریصانه به دو قسمت افراز می‌کند و در هر قسمت مسئله‌ی $1$-مرکز را جداگانه حل می‌کند. روشن است که این الگوریتم از لحاظ مرتبه‌ی زمانی بهینه است.

هرشبرگر \مرجع{hershberger1989finding} در سال ۱۹۸۹، به بررسی مسئله‌ی $2$-مرکز گسسته می‌پردازد. در نگاه اول پیداست که این مسئله دشوارتر از مسئله‌ی $1$-مرکز گسسته است و برای حل آن نیاز به روش‌های پیچیده‌تری هست. در این مقاله، هرشبرگر روش جست‌وجوی دودویی را برای یافتن مقدار $r^*$ یش می‌گیرد و موفق می‌شود که الگوریتمی با مرتبه‌ی زمانی $O(n^2)$ ارائه دهد.

در همان سال، آگاروال \مرجع{aggarwal1989finding} مسئله‌ی $1$-مرکز با نقاط پرت را در فضای $l_\infty$ بررسی می‌کند. وی الگوریتمی با زمان اجرای
$(n-z)^2 n \log n$
ارائه می‌دهد. این الگوریتم بر مبنای نمودار ورونوی مرتبه‌ی $(n-z)$~اُم در فضای $l_\infty$ طراحی شده است.

گنزالز \مرجع{gonzalez1991covering} در سال ۱۹۹۱ الگوریتمی تقریبی با زمان اجرای $O(kn)$ و ضریب تقریب $2$ برای مسئله‌ی $k$-مرکز ارائه می‌دهد و نشان می‌دهد که با فرض
$\mathrm{\mathop{NP}} \neq \mathrm{\mathop{P}}$
این ضریب تقریب، به‌ترین ضریب تقریب ممکن است. این الگوریتم به لحاظ نظری اهمیت شایانی در مطالعه‌ی الگوریتم‌های تقریبی دارد. فِدِر \مرجع{feder1988optimal} بعدها زمان اجرای این الگوریتم را به $O(n \log k)$ کاهش داد.

با بسته شدن مسئله‌ی $1$-مرکز، آگاروال \مرجع{agarwal1994planar} در سال ۱۹۹۴ الگوریتمی با مرتبه‌ی زمانی $O(n^2 \log^3 n)$ ارائه می‌دهد. می‌دانیم در هر جواب بهینه، یکی از دایره‌ها به سه نقطه یا دو نقطه‌ی متقابل محدود شده است. این الگوریتم سعی می‌کند محل بهینه‌ی مرکز را با استفاده از جست‌وجوی دودویی\پانویس{binary search} روی همه‌ی کاندیدهای ممکن، یعنی مرکز دایره‌ی محیطی\پانویس{circumcircle} همه‌ی مثلث‌ها و وسط همه‌ی پاره‌خط‌ها، پیدا کند. ولی تعداد این کاندیدها از مرتبه‌ی $\Theta(n^3)$ است و جست‌وجوی روی آن‌ها به زمان زیادی احتیاج دارد. به همین دلیل این الگوریتم از روش جست‌وجوی پارامتری \پانویس{parametric search} مگیدو \مرجع{megiddo1983applying} برای یافتن شعاع بهینه استفاده می‌کند و به این ترتیب مسئله را در زمان مطلوب حل می‌کند.

در سال ۱۹۹۴، کتز \مرجع{katz1997expander} الگوریتم دیگری برای مسئله‌ی $2$-مرکز ارائه می‌دهد که مرتبه‌ی زمانی آن مشابه مرتبه‌ی زمانی الگوریتم آگاروال است، ولی زمینه را برای پیش‌رفت‌های بعدی فراهم می‌آورد و معرف روی‌کرد جدیدی در حل مسائل هندسی می‌گردد. این الگورتیم بر مبنای گراف‌های بالنده\پانویس{expander graphs} \مرجع{alon2004probabilistic} طراحی شده است و در زمان $O(n^2 \log^3 n)$ اجرا می‌شود.

در همان سال، جارومزیک
% تلفظ؟
\مرجع{jaromczyk1994efficient} با مشاهدات جدیدی راجع به هندسه‌ی دوایر بهینه، زمان حل این مسئله را به $O(n^2 \log n)$ کاهش داد. وی مشاهده کرد که بررسی جداگانه‌ی مسئله در حالتی که دو دایره‌ی بهینه به اندازه‌ی کافی از هم دور باشند و در حالتی که دو دایره‌ی بهینه به هم نزدیک باشند، حل مسئله را آسان‌تر می‌کند. این مشاهده راه را برای پیش‌رفت‌ها بعدی باز کرد.

در سال ۱۹۹۴، اپشتاین \مرجع{eppstein1994iterated} الگوریتمی برای مسئله‌ی $1$-مرکز با نقاط پرت در فضای $l_2$ و $l_\infty$ ارائه می‌دهد. در این الگوریتم، ابتدا $\Theta(k)$ نزدیک‌ترین هم‌سایه‌های هر نقطه مشخص می‌شوند. سپس با استفاده از این اطلاعات، مسئله به $O(\frac{n}{k})$ مسئله‌ی کوچک‌تر با اندازه‌ی $O(k)$ تبدیل می‌شود. مرتبه‌ی زمانی این الگوریتم در فضای $l_2$ برابر است با
$O(n \log n + n (n-z) \log (n-z))$
و در فضای $l_\infty$ برابر است با
$O(n \log n + (n-z)^2 n)$.

افرات \مرجع{efrat1993computing} در سال ۱۹۹۵ الگوریتمی بر مبنای جست‌وجوی پارامتری برای حل همین مسئله در فضای $l_2$ ارائه می‌دهد و زمان اجرای پیشین را به
$O(n \log n + n(n-z))$
کاهش می‌دهد.

در همان سال، ماتوشک \مرجع{matouvsek1995enclosing} الگوریتمی تصادفی برای همین مسئله ارائه می‌دهد که زمان اجرای مشابهی دارد و پیچیدگی و دشواری پیاده‌سازی و ضریب ثابت پنهان معمول در الگوریتم‌های مبتنی بر جست‌وجوی پارامتری را برطرف کرده است. وی در همان سال در مقاله‌ای دیگر \مرجع{matouvsek1995geometric} الگوریتمی با زمان اجرای
$O(n \log n + (n-z)^3 n^\epsilon)$
ارائه می‌دهد که به‌ازای $z$های بزرگ عمل‌کرد به‌تری دارد. به نظر می‌رسد که برای $z$های نزدیک $\frac{n}{2}$ امیدی به به‌بود شایان این الگوریتم نیست، چرا که به‌بود آن باعث حل به‌تر برخی از مسائل $n^2$-سخت می‌شود.

در سال ۱۹۹۶، شریر \مرجع{sharir1996rectilinear} مسئله‌ی $k$-مرکز را در فضای $l_\infty$ مورد بررسی قرار می‌دهد. وی الگوریتمی خطی برای مسئله‌ی $3$-مرکز و الگوریتمی از مرتبه‌ی زمانی $O(n \log n)$ برای مسئله‌ی $4$-مرکز و الگوریتمی از مرتبه‌ی زمانی $O(n \log^5 n)$ برای مسئله‌ی $5$-مرکز ارائه می‌دهد.

شریر \مرجع{sharir1997near} در سال ۱۹۹۷، اولین الگوریتم نزدیک به خطی را برای مسئله‌ی $2$-مرکز ارائه می‌دهد. در طراحی این الگوریتم، شریر روش‌های جست‌وجوی پارامتری، جست‌وجو در ماتریس‌های یک‌نوا \پانویس{monotone} و محاسبه‌ی پویای ناحیه‌بندی دوایر را به کار می‌گیرد.

اندکی بعد از انتشار نتایج شریر، در همان سال، اپشتاین \مرجع{eppstein1997faster} الگوریتم تصادفی‌ای ارائه می‌دهد که زمان اجرای بسیار نزدیک به خطی $O(n \log^2 n)$ را به دست می‌آورد. این الگوریتم با استفاده از داده‌ساختارهای پویا زمان اجرای الگوریتم را کاهش داده و پیاده‌سازی عملی آن را آسان‌تر می‌کند.

آگاروال \مرجع{agarwal1998discrete} در سال ۱۹۹۸، با مشاهدات هندسی هوشمندانه‌ای الگوریتم هرشبرگر \مرجع{hershberger1989finding} را برای مسئله‌ی $2$-مرکز گسسته بهبود می‌دهد و زمان اجرای
$n^{\frac{4}{3}} \log^5 n$
معرفی می‌کند. این الگوریتم به‌ترین نتیجه‌ی شناخته شده برای مولفین این پایان‌نامه است و به‌بود آن مسئله جالب توجهی قلم‌داد می‌شود.

در سال ۱۹۹۹، چن \مرجع{chan1999more} با اعمال تغییراتی در الگوریتم پیش‌نهادی اپشتاین \مرجع{eppstein1997faster}، الگوریتمی قطعی با زمان اجرای نزدیک به خطی $O(n \log^2 n \log\log n)$ برای مسئله‌ی $2$-مرکز ارائه می‌دهد. زمان اجرای این الگوریتم به طور قابل اغماضی بیش‌تر از زمان اجرای الگورتیم اپشتاین است، ولی از مزیت قطعی بودن برخوردار است.

در همان سال، چن \مرجع{chan1999geometric} چهارچوبی برای طراحی الگوریتم‌های تصادفی توصیف می‌کند و به عنوان نمونه‌ای از کاربردهای آن، الگوریتمی تصادفی با مرتبه‌ی زمانی $O(n \log n)$ برای مسئله‌ی $1$-مرکز با نقاط پرت در فضای $l_\infty$ ارائه می‌دهد.

آگاروال \مرجع{agarwal2002exact} در سال ۲۰۰۲ الگوریتمی برای مسئله‌ی $k$-مرکز در فضای $d$-بعدی ارائه داد که در زمان
$n^{O(k^{1-1/d})}$
مراکز بهینه را پیدا می‌کند. به این ترتیب آگاروال ثابت کرد که اگر $k$ جزو ورودی نباشد، مسئله‌ی $k$-مرکز در زمان چندجمله‌ای قابل حل است. این الگوریتم برای حل مسئله‌ی $k$-مرکز گسسته نیز قابل استفاده است.

در سال ۲۰۰۸، آگاروال \مرجع{agarwal2008efficient} الگوریتمی تصادفی از مرتبه‌ی زمانی $n z^7 \log^3 n$ برای حل مسئله‌ی $2$-مرکز با نقاط پرت در فضای اقلیدسی ارائه می‌دهد. در این الگوریتم مسئله برای حالتی که دو دایره‌ی بهینه به اندازه‌ی کافی از هم دور باشند و حالتی که دو دایره به هم نزدیک باشند، به طول جداگانه حل می‌شود. جواب نهایی کمینه‌ی جواب هر یک از این دو حالت گزارش می‌شود. در همین مقاله، آگاروال مسئله‌ی $k$-مرکز با نقاط پرت در فضای $l_\infty$ را مورد بررسی قرار می‌دهد و برای $k=4$ الگوریتمی با زمان اجرای
$O(z^{O(1)} n \log n)$
و برای $k=5$ الگوریتمی با زمان اجرای
$O(z^{O(1)} n \log^5 n)$
ارائه می‌دهد.

آن \مرجع{ahn2011covering} در سال ۲۰۱۱ الگوریتم چن \مرجع{chan1999geometric} را به‌بود داده و مرتبه‌ی زمانی آن را به $O(n + k \log k)$ کاهش می‌دهد. برای این کار نقاطی را که وجود آن‌ها در نتیجه‌ی مسئله تاثیری ندارد کنار می‌گذارد و الگوریتم چن را روی باقی‌مانده‌ی نقاط اجرا می‌کند.
