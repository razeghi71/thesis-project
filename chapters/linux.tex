\فصل{بررسی قابلیت‌های لازم از گنو/لینوکس}

با توجه به اینکه در راه حل‌های ارائه شده برای مشکلات از قابلیت‌هایی از سیستم‌عامل گنو/لینوکس استفاده نموده‌ایم به مرور این قابلیت‌ها در زیر می‌پردازیم :

\قسمت{انتزاع دستگاه‌های ورودی و خروجی}
\برچسب{لینوکس:انتزاع}
یکی از سیاست‌های گنو/لینوکس در قبال دستگاه‌های ورودی و خروجی\پانویس{ّInput Output Devices} این است که عموماً آن‌ها را در سطح فایل منتزع میکند. یعنی اگر میخواهید در یک دستگاه خروجی (برای مثال یک سوکت\پانویس{Socket} شبکه) چیزی بنویسید کافی است تا در فایل متناظر آن دستگاه خروجی آن را بنویسید یا اگر میخواهید از یک دستگاه ورودی (برای مثال موس\پانویس{Mouse}) چیزی بخوانید کافی است فایل متناظر آن را باز کنید و از آن شروع به خواندن کنید.

البته مقررات\پانویس{Protocol} نوشتن و خواندن در دستگاه‌های مختلف متفاوت است. مثلاً اگر میخواهید از سوکتی بخوانید کافی است از فایل متناظرش، داده‌ها را بخوانید. اما اگر بخواهید از صفحه‌کلید\پانویس{Keyboard} بخوانید، باید پس از خواندن، داده‌ها را تبدیل به یک \لر{struct} کرده و سپس از آن استفاده کنید.

\قسمت{استفاده‌ی انحصاری از دستگاه‌های ورودی و خروجی}
\برچسب{لینوکس:انحصار}
در سیستم‌عامل‌های مبتنی بر گنو/لینوکس یک برنامه اگر مجوز‌های\پانویس{Permission} لازم را داشته باشد، سیستم‌عامل به او اجازه می‌دهد که یک دستگاه ورودی یا خروجی را به صورت انحصاری در اختیار بگیرد. به صورتی که هیچ برنامه‌ی دیگری قابلیت خواندن یا نوشتن در آن دستگاه را نداشت باشد. به این عمل اصطلاحاً \لر{grab}‌ کردن می‌گویند. 

شرط لازم برای انجام این کار این است که گروه موثر برنامه‌ی مورد نظر \لر{input} باشد یا کاربر \لر{root} باشد.

\قسمت{شبه‌فایل‌سیستم \لر{proc}}

در سیستم‌عامل‌های مبتنی بر گنو/لینوکس، \لر{proc} شبه‌فایل‌سیستمی\پانویس{pseudo-filesystem} است که واسطی\پانویس{interface} برای داده ساختارهای سیستم‌عامل برای نگه‌داری پردازه‌هاست و معمولاً روی \لر{/proc} سوار\پانویس{mount} می‌شود. به این صورت که هر پردازه در آن پوشه‌ای با نام شماره پردازه‌اش\پانویس{PID} داراست و درون آن فایل‌هایی وجود دارد که مشخصات آن پردازه را نشان می‌دهد.

برای مثال فایل \لر{/proc/100/exe} یک پیوند نمادین\پانویس{symlink} به فایل اجراییِ پردازه‌ای با شماره پردازه‌ی ۱۰۰ است. 

\قسمت{سوکت دامنه‌ی یونیکس}
\برچسب{لینوکس:سوکت}
سوکت دامنه‌ی یونیکس یک سوکت محلی است که معمولا در سیستم‌عامل‌های شبیه یونیکس برای ارتباط بین پردازه‌ها \پانویس{\لر{Process}} استفاده می‌شود. 

این سوکت‌ها از فایل‌سیستم به عنوان فضای‌نام خود استفاده میکنند و دو پروسه از طریق \لر{i-node}ها به این سوکت‌ها دسترسی پیدا می‌کنند و می‌توانند به یک‌دیگر پیغام بفرستند.

یکی از قابلیت‌هایی که این سوکت‌ها دارند این است که در صورتی که یک پردازه پرچم \لر {PASSCRED} را روشن کند و پردازه‌ی دیگر پرچم \لر{PEERCRED} را روشن نماید. سیستم‌عامل مشخصاتی از پردازه‌ی اول مانند شماره‌ی پردازه‌ی آن را به پردازه‌ی دوم می‌دهد و پردازه‌ی دوم می‌تواند با استفاده از این شماره پردازه و شبه‌فایل‌سیستم \لر{/proc} اطلاعاتی را در مورد پردازه‌ی اول به دست آورد.

\قسمت{مجوز اجرا با استفاده از کاربر و گروه خود فایل}
\برچسب{لینوکس:مجوز}

در گنو/لینوکس می‌توان مجوز فایل‌ها را به گونه‌ای تنظیم کرد که به جای اجرا شدن با کاربر اجرا کننده‌ی فایل، با کاربر صاحب فایل اجرا شود یا به جای اجرا شدن با گروه اجرا کننده با گروه صاحب فایل اجرا شود. این کارها به ترتیب با روشن کردن پرچم‌های \لر{suid} و \لر{sgid} برای یک فایل صورت می‌پذیرد. 

برای مثال هنگامی که شما گذرواژه‌ی کاربری را با استفاده از دستور \لر{passwd} تغییر میدهید در واقع فایل \لر{/etc/shadow} که صاحب آن کاربر \لر{root} است تغییر می‌کند و این در حالی است که شما دستور \لر{passwd} را بدون مجوز کاربر \لر{root} انجام می‌دهید. دلیل کار کردن این دستور مجوز آن است. یعنی پرچم \لر{suid} آن روشن است و به آن این امکان را میدهد که با مجوز صاحب فایل که \لر{root} است اجرا شود.


\قسمت{\لر{TTY}ها}
\برچسب{لینوکس:تی‌تی‌وای}
فایلهای \لر{/dev/tty[n]} در لینوکس فایلهایی هستند که متعلق به کاربر \لر{root} و گروه \لر{tty} می‌باشند و مجوز آنها \لر{0666} است. به این معنی که قابلیت خواندن آن‌ها وجود ندارد ولی می‌توان در آن‌ها چیزی نوشت. 

به طور معمول با کلید‌های \لر{Ctrl+Alt+F(1-7)} می‌توان بین \لر{tty}های ۱ تا ۷ جابجا شد (روی این ttyها به صورت پیشفرض برنامه‌ی \لر{login} اجراست). اگر نیاز داریم تا به \لر{tty} باشماره‌ی بالاتر برویم میتوانیم با دستور \کد{chvt [tty num]} این کار را انجام دهیم.

همچنین برای اجرای یک برنامه بر روی یک \لر{tty} می‌توانیم از ابزار \لر{openvt} استفاده نماییم.



