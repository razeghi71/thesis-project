\فصل{نرم‌افزار مدیریت گذرواژه و قابلیت‌های آن}

مدیر گذرواژه نرم‌افزاری کامپیوتری است که به کاربر کمک می‌کند که گذرواژه‌هایش را ذخیره و ساماندهی کند. مدیر‌های گذرواژه عموماً گذرواژه‌ها را به صورت رمزنگاری شده ذخیره می‌کنند. 

این نرم‌افزارها برای رمزنگاری گذرواژه‌ها از یک گذرواژه‌ که به اندازه‌ی کافی قوی است استفاده میکنند و آن را گذرواژه‌ی اصلی می‌نامند. معمولاً گذرواژه‌ی اصلی هنگام نصب نرم‌افزار از کاربر گرفته می‌شود(فرض می‌شود که در هنگام نصب کلیددزدی بر روی سیستم نصب نیست) و با توجه به اینکه ذخیره‌ی خود آن کار خطرناکی است \لر{Hash} آن ذخیره می‌شود. هر موقع که برنامه‌ی مدیر گذرواژه اجرا می‌شود کاربر باید این گذرواژه را وارد کند تا بتواند سایر گذرواژه‌ها را ببیند.

بعضی از مدیر‌های گذرواژه، عملیات ذخیره‌سازی گذرواژه‌ها را بر روی کامپیوتر محلی انجام می‌دهند در حالی که بعضی دیگر این گذرواژه‌ها را روی یک فضای ابری ذخیره‌سازی می‌کنند. گونه‌ای دیگر از مدیر‌های گذرواژه  به کاربر این قابلیت را می‌دهند که گذرواژه‌ها را روی فضای ابری خودش ذخیره‌سازی کند و نه فضای ابری سرویس‌دهنده\پانویس{Provider}.

بسیاری از نرم‌افزار‌های مدیریت گذرواژه علاوه بر ذخیره‌سازی گذرواژه‌ها، خدمات دیگری نیز انجام می‌دهند که در ادامه به اختصار به توضیح چند مورد از آن‌ها میپردازیم.

\قسمت{ارائه‌ی واسط برنامه‌نویسی به نرم‌افزارهای دیگر}

امروزه بسیاری از نرم‌افزارها در کارهایشان نیاز به احراز هویت دارند و این احراز هویت عموماً توسط شناسه‌ی کاربری و گذرواژه انجام می‌شود. 

یک مدیر گذرواژه می‌تواند با ارائه‌ی واسط برنامه‌نویسی به دیگر برنامه‌ها این قابلیت را به آن‌ها بدهد که آن‌ها شناسه‌ی کاربری\پانویس{username} و گذرواژه خود را ذخیره‌سازی و بعداً بدون نیاز به پرسیدن مجدد از کاربر بازیابی کند.

برای مثال یک مرورگر وب می‌تواند افزونه‌ای\پانویس{Extension} بنویسد که با اتصال به نرم‌افزار مدیر گذرواژه، نام‌کاربری و رمز‌های عبور یک وب‌سایت را ذخیره‌سازی کند و دفعه‌ی بعدی که کاربر به آن وب‌سایت مراجعه نمود به صورت خودکار آن فیلد‌ها را پر کند. 

یا به عنوان مثال برای یک نرم‌افزار دسکتاپ میتوان برنامه‌ی \لر{sudo} را نام برد که هر دفعه گذرواژه‌ی کاربر را می‌پرسد که برنامه‌ای را با کاربر موثر \لر{root} اجرا کند. میتوان برنامه‌ای مانند \لر{ssudo} نوشت که اولین بار با گرفتن گذرواژه از کاربر آن را در نرم‌افزار مدیر گذرواژه ذخیره کند و از دفعات بعدی بدون گذرواژه کار را انجام دهد.

\قسمت{تولید رمز عبور امن}

با توجه به اینکه رمزهای عبٖوری که کاربران انتخاب می‌کنند عموماً ناامن است، نرم‌افزارهای مدیریت گذرواژه پیشنهاد می‌دهند که علاوه بر ذخیره‌سازی گذرواژه‌ها بخش تولید گذرواژه را نیز به عهده‌ی آنها بگذاریم. به این صورت آن‌ها گذرواژه‌های امنی را برای کاربران تولید و ذخیره میکنند و کاربران می‌توانند بدون حتی به خاطر سپردن آن کلمه‌ی عبور بر روی دستگاهی که مدیر کلمه‌ی عبور وصل است احراز هویت نمایند.

\قسمت{به اشتراک گذاری امن گذرواژه}

بسیاری از اوقات ما نیازمندیم تا یک گذرواژه را با یکی از دوستانمان به اشتراک بگذاریم. اما انجام این کار روی یک کانال ناامن می‌تواند سبب لو رفتن آن گذرواژه شود. بسیاری از نرم‌افزارهای مدیریت گذرواژه قابلیت به اشتراک گذاری امن گذرواژه با دوستان و آشنایان را فراهم می‌آورند.

\قسمت{پشتیبانی از چندین پایگاه‌داده با گذرواژه‌ی اصلی متفاوت}

بعضی از نرم‌افزارهای مدیریت گذرواژه این قابلیت را به کاربر می‌دهند که بتوانند گذرواژه‌های خود را در پایگاه‌داده‌های مختلف با گذرواژه‌ی اصلی متفاوت ذخیره کنند. کاربرانی که از این قابلیت استفاده میکنند کمتر در خطر دزدیده شدن تمامی گذرواژه‌هایشان هستند.
