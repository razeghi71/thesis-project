
\فصل{طراحی نرم‌افزار}

\قسمت{زیر‌سامانه‌ها}

با توجه به توضیحات داده شده در مورد نرم‌افزار‌های مدیر گذرواژه و راه‌حل‌های ارایه شده، نرم‌افزار طراحی شده توسط ما شامل زیرسامانه‌های زیر است :

\شروع{شمارش}

\فقره{زیرسامانه‌ی \لر{Connection}}

این زیر سامانه، کلاس‌های لازم برای برقراری ارتباط بین پردازه‌ای همراه با ارسال \لر{Credential} را فراهم می‌آورد. 

\فقره{زیرسامانه‌ی \لر{VKeyboard}} 

این زیرسامانه، کلاس‌های لازم جهت خواندن از دستگاه خام صفحه کلید واقع در شبه‌فایل‌سیستم \لر{/dev} و جداسازی آن از /لر{X Server} توسط /لر{grab} کردن را انجام می‌دهد. همچنین کلاس‌های مربوط به نگاشت بین کلید‌ها و حرف متناظر آن‌ها نیز در اینجا قرار دارند.

\فقره{زیرسامانه‌ی \لر{DB}}

این زیرسامانه انتزاع لازم جهت کار کردن با پایگاه داده را فراهم می‌آورد. 

\فقره{زیرسامانه‌ی \لر{Encryption}}

این زیرسامانه با استفاده از کتابخانه‌ی \لر{Crypto++} انتزاع لازم جهت علیات رمز‌نگاری را فراهم می‌آورد.

\فقره{زیرسامانه‌ی \لر{Application‌Runner}}

این زیرسامانه کارهای لازم جهت اجرای یک برنامه‌ی خارجی مانند \لر{PasswordGetter} را به صورت‌های مختلف (مثلاً با مجوز \لر{root} یا با \لر{openvt} روی یک \لر{tty}) انجام می‌دهد.

\فقره {زیرسامانه‌ی \لر{PasswordGetter}}

این زیرسامانه یک برنامه‌ی خارجی است که وظیفه‌ی گرفتن گذرواژه‌ی اصلی را از کاربر دارد. 

\فقره{زیرسامانه‌ی \لر{PasswordChecker}}

این زیرسامانه مسئولیت خواندن کلمه‌ی احراز اصالت برنامه و چک کردن درست بودن گذرواژه‌ی اصلی را به عهده دارد.

\فقره{زیرسامانه‌ی \لر{APIListener}}

این زیرسامانه وظیفه‌ی مدیریت واسط برنامه نویسی برای برنامه‌های خارجی که قصد مدیریت گذرواژه ها را دارند داراست.

\فقره{زیرسامانه‌ی \لر{APIMessageViewer}}

این زیرسامانه مسئولیت نشان دادن پیغام‌های زیرسامانه‌ی قبلی و گرفتن تایید را داراست.

\فقره{زیرسامانه‌ی \لر{Main}}

این زیرسامانه برنامه‌ی اصلی که اجرا می‌شود است و سایر زیر‌سامانه‌ها را به هم متصل می‌کند.

\پایان{شمارش}

\قسمت{کتابخانه‌ها و چارچوب‌های استفاده شده در برنامه}
\برچسب{نرم‌افزار:نیازمندی}
\شروع{شمارش}

\فقره{چارچوب \لر{Qt}}

این چارچوب کتابخانه‌هایی برای کار ساده‌تر با رشته‌ها، بایت‌ها، برنامه‌نویسی رویداد محور\پانویس{Event Driven} با سوکت دامنه‌ی یونیکس و همچنین توابع درهم ساز را در اختیار می‌گذارد. 

\فقره{کتابخانه‌ی \لر{odb}} 

این کتابخانه، کار کردن شی‌گرا\پانویس{Object Oriented} با پایگاه‌داده‌های رابطه‌ای\پانویس{Relational} را در زبان \لر{C++} میسر می‌سازد.

\فقره{کتابخانه‌ی \لر{Crypto++}}

پیاده‌سازی توابع رمزنگاری استفاده شده برای رمز کردن گذرواژه‌ها در این کتاب‌خانه قرار دارد.

\پایان{شمارش}

\قسمت{طریقه‌ی کامپایل برنامه}

برای کامپایل برنامه‌های \لر{C++} معمولاً 	از سیستم سازنده\پانویس{Build System} \لر{Makefile} استفاده می‌شود. ولی امروزه با توجه به اینکه \لر{Makefile}ها معمولاً بعد از گذشت زمان دیگر قابلیت نگه‌داری\پانویس{Maintainability} ندارند از سیستم‌های سازنده‌ی سطح بالاتر که \لر{Makefile} تولید می‌کنند استفاده می‌شود.

با توجه به اینکه ما در این پروژه از چارچوب برنامه‌نویسی \لر{Qt} بهره بردیم، لذا از سیستم سازنده‌ی اختصاصی آن که \لر{qmake} است نیز استفاده کردیم.

برای کامپایل برنامه، ابتدا مواردی که در بخش \ارجاع{نرم‌افزار:نیازمندی} مطرح شده را نصب کنید. سپس در فایل محتوی پروژه پوشه‌ای مثلاً به نام \لر{build} بسازید. حال در یک ترمینال به این پوشه رفته و دستورات زیر را بزنید :

\lstset{basicstyle=\footnotesize\ttfamily,breaklines=true,language=Bash}
\begin{latin}
\begin{lstlisting}
$ qmake ..
$ make
\end{lstlisting}
\end{latin}

فایل اجرایی اصلی برنامه در پوشه‌ی \لر{Main} با نام \لر{Main} ساخته می‌شود.

\قسمت{توضیح واسط برنامه‌نویسی}

شرایط وصل شدن به واسط برنامه‌نویسی به شرح زیر است :

\شروع{شمارش}
\فقره فایل اجرایی مشتری نباید مجوز نوشتن توسط کاربر عادی را داشته باشد.
\فقره برنامه‌ی مشتری باید با یک سوکت دامنه‌ی یونیکس به آدرس \لر{/tmp/passman\_api.sock} وصل شود. همچنین باید پرچم \لر{SO\_PASSCRED} را برای ارسال \لر{Credential} روشن کند.
\پایان{شمارش}

پس از وصل شدن، برنامه‌ی مشتری می‌تواند دستورات زیر را به مدیر گذرواژه ارسال کند :

\شروع{شمارش}
\فقره{دستور \لر{set}}

این دستور وظیفه‌ی ذخیره‌سازی شناسه‌ی کاربری و گذرواژه‌ی جدید را دارد. طریقه‌ی کار این دستور به شکل زیر است (منظور از \لر{C} مشتری و منظور از \لر{S} کارگذار است) : 

\lstset{basicstyle=\footnotesize\ttfamily,breaklines=true}
\begin{latin}
\begin{lstlisting}
C : set
C : username
C : password
S : 6
\end{lstlisting}
\end{latin}

به این صورت که مشتری ابتدا در یک خط دستور \لر{set} و در دو خط بعدی شناسه‌ی کاربری و گذرواژه را می‌فرستد سپس کارگذار پس از ذخیره‌سازی گذرواژه، شناسه‌ی گذرواژه‌ی ذخیره شده را باز می‌گرداند.

\فقره{دستور \لر{get}}

این دستور وظیفه‌ی بازیابی گذرواژه‌ی از پیش ذخیره شده را دارد. طریقه‌ی کار این دستور به شکل زیر است (منظور از \لر{C} مشتری و منظور از \لر{S} کارگذار است) : 

\lstset{basicstyle=\footnotesize\ttfamily,breaklines=true}
\begin{latin}
\begin{lstlisting}
C : get
C : 6
S : username
S : password
\end{lstlisting}
\end{latin}

به این صورت که مشتری ابتدا در یک خط دستور \لر{get} و در خط بعدی شناسه‌ی گذرواژه‌ی از پیش ذخیره شده را می‌فرستد سپس کارگذار در دو خط شناسه‌ی کاربری و گذرواژه‌ی ذخیره‌ شده را باز می‌گرداند.

\پایان{شمارش}
