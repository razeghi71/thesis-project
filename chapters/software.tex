\فصل{طراحی نرم‌افزار}

\قسمت{زیر‌سامانه‌ها}

با توجه به توضیحات داده شده در مورد نرم‌افزارهای مدیر گذرواژه و راه‌حل‌ها برای رفع مشکلات، نرم‌افزار طراحی شده شامل زیرسامانه‌های زیر است :

\شروع{شمارش}

\فقره{زیرسامانه‌ی \لر{Connection}}

این زیر سامانه، کلاس‌های لازم برای برقراری ارتباط بین پردازه‌ای همراه با \لر{Credential} را فراهم می‌آورد. 

\فقره{زیرسامانه‌ی \لر{VKeyboard}} 

این زیرسامانه، کلاس‌های لازم جهت خواندن از دستگاه خام صفحه کلید واقع در شبه‌فایل‌سیستم \لر{/dev} و جداسازی آن از /لر{X Server} توسط /لر{grab} کردن را انجام می‌دهد. همچنین کلاس‌های مربوط به نگاشت بین کلید‌ها و حرف متناظر آن‌ها نیز در اینجا قرار دارند.

\فقره{زیرسامانه‌ی \لر{DB}}

این زیرسامانه انتزاع لازم جهت کار کردن با پایگاه داده را فراهم می‌آورد. 

\فقره{زیرسامانه‌ی \لر{Encryption}}

این زیرسامانه با استفاده از کتابخانه‌ی \لر{Crypto++} انتزاع لازم جهت علیات رمز نگاری را فراهم می‌آورد.

\فقره{زیرسامانه‌ی \لر{Application Runner}}

این زیرسامانه کارهای لازم جهت اجرای یک برنامه‌ی خارجی مانند \لر{PasswordGetter} را انجام می‌دهد. 

\فقره {زیرسامانه‌ی \لر{PasswordGetter}}

این زیرسامانه یک برنامه‌ی خارجی است که وظیفه‌ی گرفتن گذرواژه‌ی اصلی را از کاربر دارد.

\فقره{زیرسامانه‌ی \لر{PasswordChecker}}

این زیرسامانه مسئولیت خواندن کلمه‌ی احراز هویت و چک کردن درست بودن گذرواژه‌ی اصلی را به عهده دارد.
لپش
\فقره{زیرسامانه‌ی \لر{APIListener}}

این زیرسامانه وظیفه‌ی مدیریت واسط برنامه نویسی برای برنامه‌های خارجی که قصد مدیریت گذرواژه ها را دارند داراست.

\فقره{زیرسامانه‌ی \لر{APIMessageViewer}}

این زیرسامانه مسئولیت نشان دادن پیغام‌های زیرسامانه‌ی قبلی و گرفتن تایید را داراست.

\فقره{زیرسامانه‌ی \لر{Main}}

این زیرسامانه برنامه‌ی اصلی که اجرا می‌شود است و سایر زیر‌سامانه‌ها را به هم متصل می‌کند.

\پایان{شمارش}

