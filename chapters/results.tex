\فصل{فضای بیشین}
\برچسب{chapter:l_inf}
در این فصل به مطالعه‌ی مسئله‌ی $k$-مرکز در فضای $l_\infty$ می‌پردازیم. یادآوری می‌کنیم که در فضای $l_\infty$ فاصله‌ی دو نقطه طبق رابطه‌ی
$\max\set{\card{x_1 - x_2}, \card{y_1 - y_2}}$
محاسبه می‌گردد. هدف نهایی این فصل ارائه‌ی راه‌حلی برای مسئله‌ی $1$-مرکز با نقاط پرت وزن‌دار در فضای $l_\infty$ دوبعدی است. تعریف دقیق این مسئله به شرح زیر است.
\شروع{مسئله}
\برچسب{مسئله:تک‌مرکز با نقاط پرت وزن‌دار}
مجموعه‌ی $n$ نقطه‌ی وزن‌دار $\mathcal{C}$ در فضای $l_\infty$ دوبعدی و عدد حقیقی $0 \leq w$ داده شده‌اند. هدف پیدا کردن کوچک‌ترین گوی با مرکز روی یکی از $n$ کاندید $\mathcal{F}$ است که وزن نقاط خارج از آن حداکثر برابر $w$ باشد.
\پایان{مسئله}

پیش از این که به بیان راه‌حل پیش‌نهادی این مسئله بپردازیم، مسئله‌ی ساده‌تری مطرح می‌کنیم که پیش‌تر توسط سایر محققان بررسی و حل شده است.

\قسمت{تک‌مرکز}
\شروع{مسئله}
\برچسب{مسئله:۱-مرکز}
مجموعه‌ی نقاط $\mathcal{P}$ در صفحه‌ی دوبعدی داده شده‌اند. هدف پیدا کردن کوچک‌ترین مربع با اضلاع موازی محورهای مختصات است که تمامی نقاط را بپوشاند.
\پایان{مسئله}

یکی از خواص ویژه‌ی توابع اندازه‌ی مربعی، یعنی
$l_1$
و
$l_\infty$،
متناهی بودن تعداد جهات عمود بر گوی واحد است. این ویژگی حل مسئله‌ی $1$-مرکز را بسیار آسان می‌کند. برای این کار کافی است که در هر جهت، نقاط مفرط را پیدا کنیم و به این ترتیب مکعب مستطیل محیطی مجموعه‌ی نقاط را تعیین کنیم. نصف طول بلندترین ضلع این مکعب مستطیل، شعاع بهینه‌ی تک‌مرکز خواهد بود و گوی بهینه نیز، کوچک‌ترین گوی شامل این مکعب است. به این ترتیب می‌توان این مسئله را در زمان خطی حل کرد. به عنوان مثال در شکل~\رجوع{fig:l_inf 1-center example} یک گوی بهینه و نقاط مفرط مشخص شده‌اند.

\شروع{شکل}[h]
\تنظیم‌ازوسط
\درج‌تصویر[width=7cm]{linf_1center_exmpl}
\شرح{مثالی از مسئله‌ی $1$-مرکز در فضای $l_\infty$}
\برچسب{fig:l_inf 1-center example}
\پایان{شکل}


\قسمت{تک مرکز با نقاط پرت}

\شروع{مسئله}
\برچسب{مسئله:۱-مرکز با نقاط پرت}
مجموعه‌ی نقاط $\mathcal{P}$ در صفحه‌ی دوبعدی و عدد صحیح $0 \leq z \leq n$ داده شده‌اند. هدف پیدا کردن کوچک‌ترین مربع با اضلاع موازی محورهای مختصات است که تمامی نقاط به جز حداکثر $z$ نقطه را بپوشاند.
\پایان{مسئله}

چن \cite{chan1999geometric} چهارچوبی برای تعریف الگوریتم‌های تصادفی ارائه داده و توسط آن به حل این مسئله پرداخته است. در ادامه راه‌حل پیش‌نهادی چن را بیان می‌کنیم.

\شروع{قضیه}
\برچسب{قضیه:چن}
فرض کنید $\Pi$ فضای مسئله باشد. ارزش جواب یک مسئله مانند $P$ را با $s(P)$ و اندازه‌ی ورودی آن را با $\card{P}$ نمایش می‌دهیم. فرض کنید اعداد حقیقی $\alpha < 1$ و $\epsilon > 0$ و عدد صحیح و مثبت $k$ و تابع $D(\cdot)$ وجود داشته باشند که
$\frac{D(n)}{n^\epsilon}$
نسبت به $n$ صعودی باشد. فرض کنید الگوریتم از مرتبه‌ی زمانی متوسط $O(D(\card{P}))$ وجود داشته باشد که:
\شروع{فقرات}
\فقره به ازای هر $t \in \IR$ تصمیم بگیرد که آیا $t < s(P)$ یا خیر.
\فقره $k$ مسئله‌ی
$P_1, P_2, \ldots, P_k \in \Pi$
با اندازه‌ی حداکثر
$\ceil{\alpha\card{P}}$
را بسازد طوری که $s(P)$ کمینه‌ی مقدار $s(P_i)$~ها باشد.
\پایان{فقرات}
در این صورت برای هر $P \in \Pi$ مقدار $s(P)$ در زمان متوسط
$O(D(\card{P}))$
قابل محاسبه است.
\پایان{قضیه}

\شروع{اثبات}
پیش از تحلیل این قضیه به الگوریتم~\رجوع{الگوریتم:کمینه} توجه کنید. این الگوریتم با گرفتن یک دنباله‌ی $r$~تایی به عنوان ورودی، کمینه‌ی عناصر آن را محاسبه می‌کند.
\begin{algorithm}
\caption{پیدا کردن کمینه}
\label{الگوریتم:کمینه}
\begin{algorithmic}[1]
\Function{کمینه}{{$(a_i)_{i=1}^r$}}
\State{یک جای‌گشت تصادفی مانند $(i_j)_{j=1}^r$ از $1, \ldots, r$ را انتخاب کن.}
\State{$\infty \rightarrow t$}
\For{$1, \ldots, r \rightarrow j$}
	\If{$a_{i_j} < t$}
		\State{$a_{i_j} \rightarrow t$}\label{خط:مقدار کمینه‌ی جدید}
	\EndIf
\EndFor
\State\Return{$t$}
\EndFunction
\end{algorithmic}
\end{algorithm}

اجرای $j$~اُم حلقه را در این االگورتیم در نظر بگیرید. احتمال اجرای خط~\رجوع{خط:مقدار کمینه‌ی جدید} در این اجرا برابر $\frac{1}{j}$ است. پس متوسط تعداد دفعات اجرای این خط برابر است با:
\[
E[\text{ تعداد اجرا }]
= \sum_{j=1}^k P[\text{ احتمال اجرای خط در گردش $j$~اُم حلقه }]
= \sum_{j=1}^k \frac{1}{j}
= H_k
\leq ln(k) + 1
\]

این لم اساس اثبات قضیه‌ی چن است. به الگوریتم~{الگوریتم:بهینه} توجه کنید. این الگورتیم برای محاسبه‌ی جواب بهینه‌ی مسئله‌ی $P$ طراحی شده است. در این‌جا فرض شده است که مسئله‌ی $P$ در شرایط قضیه‌ی چن صدق می‌کند.

\begin{algorithm}
\caption{پیدا کردن جواب بهینه در شرایط قضیه‌ی چن}
\label{الگوریتم:بهینه}
\begin{algorithmic}[1]
\Function{بهینه}{$P$}
\State{زیرمسئله‌های کوچک‌تر $P_1, \ldots, P_k$ را تولید کن.}	\label{خط:ایجاد زیرمسئله‌ها}
\State{یک جای‌گشت تصادفی مانند $(i_j)_{j=1}^k$ از $1, \ldots, k$ را انتخاب کن.}
\State{$\infty \rightarrow t$}
\For{$1, \ldots, k \rightarrow j$}
	\If{$s(P_{i_j}) < t$}	\label{خط:مقایسه با مقدار بهینه‌ی فعلی}
		\State{$s(P_{i_j}) \rightarrow t$}	\label{خط:مقدار بهینه‌ی جدید}
	\EndIf
\EndFor
\State\Return{$t$}
\EndFunction
\end{algorithmic}
\end{algorithm}

این الگوریتم با مقایسه‌ی ارزش جواب زیرمسئله‌ها، مقدار بهینه‌ی جواب مسئله‌ی $P$ را محاسبه می‌کند. برای این کار در خط~\رجوع{خط:ایجاد زیرمسئله‌ها} مسئله‌ی $P$ را به زیرمسئله‌های $P_1, \ldots, P_k$ تقسیم می‌کند طوری که مقدار بهینه‌ی جواب مسئله‌ی $P$ برابر کمینه‌ی ارزش جواب مسئله‌های $P_1, \ldots, P_k$ باشد. طبق فرض، در زمان متوسط $O(D(\card{P}))$ می‌توان $P_1, \ldots, P_k$ را تولید کرد. همچنین طبق فرض،‌اجرای خط~\رجوع{خط:مقایسه با مقدار بهینه‌ی فعلی} نیز در زمان متوسط $O(D(\card{P}))$ قابل محاسبه است. طبق آن‌چه در الگوریتم~\رجوع{الگوریتم:کمینه} دیدیم، خط~\رجوع{خط:مقدار بهینه‌ی جدید} به طور متوسط $\ln k$ بار اجرا می‌شود. در هر بار اجرای آن، الگوریتم به طور بازگشتی روی مسئله‌ی کوچک‌تر $P_{i_j}$ اجرا می‌شود. پس اگر زمان متوسط اجرای این الگوریتم را بر حسب اندازه‌ی ورودی با $T(\cdot)$ نمایش دهیم:
\[
T(n) \leq O(D(n)) +‌ O(n) + k O(D(\card{P})) + (\ln(k)+1) T(\ceil{\alpha n})
\]
طبق قضیه‌ی اصلی، اگر
$(\ln(r) + 1) \alpha^\epsilon < 1$
آن‌گاه الگوریتم در زمان متوسط
$O(D(n) + n)$
اجرا می‌شود. می‌توان دید که شرط
$(\ln(r) + 1) \alpha^\epsilon < 1$
شرط سخت‌گیرانه‌ای نیست و همیشه می‌توان با فشرده کردن طبقات درخت اجرای این الگوریتم، این شرایط را محیا کرد.
\پایان{اثبات}


\شروع{قضیه}
مسئله‌ی~\رجوع{مسئله:۱-مرکز با نقاط پرت} به طور متوسط در زمان $O(n \log n)$ قابل حل است.
\پایان{قضیه}

\شروع{اثبات}
در اثبات این ادعا از قضیه‌ی~\رجوع{قضیه:چن} استفاده می‌کنیم. نشان می‌دهیم می‌توان در زمان $O(n \log n)$ مسئله را به $5$ زیرمسئله با اندازه‌ی حداکثر $\ceil{\frac{4}{5} n}$ تقسیم کرد و همچنین می‌توان در زمان $O(n \log n)$ تصمیم گرفت که آیا عدد داده شده‌ی $t$ بیش‌تر از جواب بهینه است یا خیر.
در واقع سعی می‌کنیم با استفاده از این چهارچوب مسئله‌ی سخت‌تر زیر را حل کنیم:
\شروع{مسئله}
مجموعه‌ی $n$ نقطه‌ی $\mathcal{P}$ در فضای $l_\infty$ دوبعدی، عدد صحیح $0 \leq z \leq n$ و مستطیل $R$ داده شده‌اند. هدف پیدا کردن کوچک‌ترین گوی شامل $R$ و شامل $\mathcal{P}$ به جز حداکثر $z$ نقطه از آن است.
\پایان{مسئله}
\برچسب{مسئله:تعمیم ۱-مرکز با نقاط پرت}
برای تصمیم گیری $s(P) < t$ می‌توان از الگوریتم اپشتاین \مرجع{eppstein1994iterated} استفاده کرد و این تصمیم‌گیری را در زمان مطلوب انجام داد. پس کافی است نحوه‌ی تقسیم مسئله به زیرمسئله‌ها را شرح دهیم. نیم‌صفحه‌های $H_1, \ldots, H_4$ را در نظر بگیرید که در هر یک از چهار جهت $\ceil{\frac{n}{5}}$ نقاط را جدا می‌کنند. مستطیل حاصل از اشتراک این چهار نیم‌صفحه را با $R_0$ نمایش می‌دهیم. شکل~\رجوع{شکل:چن} نمونه‌ای از این تقسیم‌بندی را نمایش می‌دهد.
\شروع{شکل}
\تنظیم‌ازوسط
\درج‌تصویر
[width=7cm]{chan}
\شرح{نمونه‌ای از تقسیم‌بندی راه‌حل مسئله‌ی~\رجوع{مسئله:تعمیم ۱-مرکز با نقاط پرت}}
\برچسب{شکل:چن}
\پایان{شکل}

روشن است که مربع بهینه یا کاملاً در یکی از این چهار نیم‌صفحه قرار دارد، یا کاملاً مستطیل $R_0$ را در بر می‌گیرد. همچنین مشخص است که مستطیل $R_0$ شامل حداقل $\frac{n}{5}$ نقاط است. پس به ازای هر یک از چهار نیم‌صفحه‌ی $H_i$ زیرمسئله‌ی
$(\mathcal{P}_i, R_i, z_i)$
را برابر
$(\mathcal{P} \cap H_i, R, z - \card{\mathcal{P} - H_i}($
تعریف می‌کنیم و زیرمسئله‌ی پنجم را به صورت
$(\mathcal{P} - R_0, R \vee R_0, z)$
تعریف می‌کنیم. روشن است که اگر مربع بهینه در یکی از چهار نیم‌صفحه باشد، یکی از چهار زیرمسئله‌ی اول جواب مسئله را مشخص می‌کند و اگر مربع بهینه به طور کامل در هیچ یک از چهار نیم‌صفحه جای نگیرد زیرمسئله‌ی پنجم جواب مسئله را مشخص می‌کند. اندازه‌ی هر یک از این چهار زیرمسئله حداکثر
$\ceil{\frac{4n}{5}}$
است. برای جدا کردن نقاط در هر یک از چهار جهت کافی است نقاط را برحسب دو مولفه مرتب کرده باشیم که این کار در زمان $O(n \log n)$ قابل انجام است. بنابراین چهارچوب چن الگوریتمی تصادفی با زمان متوسط $O(n \log n)$ برای حل این مسئله ارائه می‌دهد.
\پایان{اثبات}

شایان ذکر است که دیمین \مرجع{ahn2011covering} با در نظر گرفتن نقاط $z$-مفرط در هر جهت و نادیده گرفتن سایر نقاط، زمان اجرای این الگوریتم را به $O(n + z \log z)$ کاهش داده است.

در ادامه ادعا می‌کنیم این الگوریتم از لحاظ زمان اجرا با به‌ترین کران پایین آن تفاوت چندانی ندارد. برای این کار از نتیجه‌ی~\رجوع{نتیجه:کران پایین برای تمایز عناصر} که در مقدمه معرفی شده است، استفاده می‌کنیم. در این حکم دیدیم که حل مسئله‌ی تمایز عناصر یک دنباله در زمان کم‌تر $\Omega(n \log n)$ ممکن نیست. در این‌جا نشان می‌دهیم مسئله‌ی تمایز عناصر به مسئله‌ی $1$-مرکز با نقاط پرت در زمان خطی قابل کاهش است. توجه کنید که عناصر یک مجموعه از نقاط متمایز هستند اگر و تنها اگر نتوان هیچ دو تایی از آن‌ها را با گویی به شعاع صفر پوشاند. یعنی شعاع $1$-مرکز آن (با هر تابع فاصله‌ای) با $n-2$ نقطه‌ی پرت بیش‌تر از صفر باشد. این کاهش نشان می‌دهد که حل مسئله‌ی $1$-مرکز برای در هر فضایی تحت مدل درخت محاسبه‌ی جبری به حداقل $\Omega(n \log n)$ زمان نیاز دارد.

\شروع{قضیه}
تحت مدل محاسباتی درخت محاسبه‌ی جبری، هیچ الگوریتمی با زمان اجرای به‌تر از $O(n \log n)$ برای مسئله‌ی~\رجوع{مسئله:۱-مرکز} وجود ندارد.
\پایان{قضیه}


\قسمت{تک‌مرکز با نقاط پرت ‌وزن‌دار}

در این قسمت الگوریتم پیش‌نهادی خود را برای مسئله‌ی~\رجوع{مسئله:تک‌مرکز با نقاط پرت وزن‌دار} تشریح می‌کنیم. الگوریتم پیش‌نهادی ما بر چهارچوب چن در قضیه‌ی~\رجوع{قضیه:چن} استوار است. این الگوریتم در زمان متوسط $O(n \log n)$ اجرا می‌شود. طبق قضیه‌ی~\رجوع{قضیه:چن}، کافی است که الگوریتمی با مرتبه‌ی زمانی متوسط $O(n \log n)$ برای تقسیم مسئله به زیرمسئله‌ها و الگوریتمی با همین زمان متوسط برای تصمیم‌گیری $s(P) < t$ ارائه کنیم. می‌توان دید که تقسیم مسئله به زیرمسئله دقیقاً مشابه الگوریتم پیش‌نهادی چن قابل انجام است. پس کافی است که الگوریتمی برای تصمیم‌گیری $s(P) < t$ طراحی کنیم.

برای این کار از داده‌ساختار درخت محدوده‌ای استفاده می‌کنیم. ساختن این درخت در زمان $O(n \log n)$ قابل انجام است. در هر گره از درخت، مجموع وزن نقاط مربوط آن گره را ذخیره می‌کنیم. برای پاسخ به پرسش $s(P) > t$، مربعی به ضلع $2t$ حول هر یک از نقاط $\mathcal{F}$ در نظر می‌گیریم و در هر یک از این مربع مجموع وزن نقاط پوشیده شده را به وسیله‌ی درخت جست‌وجوی محدوده‌ای محاسبه می‌کنیم. اگر بیشینه‌ی این مجموع‌ها کم‌تر
$\sum_{c \in \mathcal{C}} w_c - t$
بود یعنی گویی با این شعاع نمی‌توان جواب مسئله باشد. پس $s(P) > t$. هر یک از پرسش‌ها برای درخت جست‌وجوی محدوده‌ای در زمان $O(\log \card{\mathcal{C}})$ قابل پاسخ‌گویی است و تعداد این پرسش‌ها برابر $\card{\mathcal{F}}$ است. پس مسئله‌ی تصمیم‌گیری $s(P) > t$ در زمان $O(n \log n)$ قابل حل است.

به این ترتیب طبق قضیه‌ی~\رجوع{قضیه:چن} مسئله‌ی $1$-مرکز با نقاط پرت ‌وزن‌دار در زمان متوسط $O(n \log n)$ قابل حل است. با توجه به این که این مسئله تعمیمی بر مسئله‌ی~\رجوع{مسئله:۱-مرکز با نقاط پرت} است، پس کران پایین $\Omega(n \log n)$ برای این مسئله نیز قابل اثبات است.
