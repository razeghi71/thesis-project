
% -------------------------------------------------------
%  Abstract
% -------------------------------------------------------


\pagestyle{empty}

\شروع{وسط‌چین}
\مهم{چکیده}
\پایان{وسط‌چین}
\بدون‌تورفتگی
نرم‌افزارهای نیازمند احراز هویت روز به روز در حال افزایش هستند و با توجه به اینکه انتخاب گذرواژه‌ای یکسان برای تمامی این نرم‌افزارها از لحاظ امنیتی کاری نادرست به حساب می‌آید نیاز به نرم‌افزارهای مدیریت گذرواژه بیشتر و بیشتر حس می‌شود. این نرم‌افزارها قابلیت‌هایی برای ساماندهی گذرواژه‌ها مانند رمزنگاری آنها، ارایه‌ی واسط‌های برنامه‌نویسی، تولید گذرواژه و ... را پشتیبانی می‌کنند.

امروزه یکی از مشکلاتی که نرم‌افزارهای مدیریت گذرواژه را تهدید می‌کند، مقاوم نبودن در مقابل کلیددزدهاست. به این معنی که اگر کلیددزدی بر روی سیستم شما نصب باشد و گذرواژه‌ی اصلی (که در واقع کلید دسترسی به بقیه‌ی گذرواژ‌ه‌هاست) را به دست آورد، می‌تواند به تمامی گذرواژه‌های شما دسترسی پیدا کند.
 
در این پایان‌نامه روشی ارایه شده است که طی آن می‌توان نرم‌افزار مدیریت گذرواژه‌ای برای سیستم‌های مبتنی بر گنو/لینوکس ساخت که در مقابل کلید‌دزد و دیگر خطر‌های احتمالی تا حد امکان مقاوم باشد.
 
\پرش‌بلند
\بدون‌تورفتگی \مهم{کلیدواژه‌ها}: 
نرم‌افزار مدیریت گذرواژه، کلید‌دزد، امنیت داده، احراز هویت، گنو/لینوکس
\صفحه‌جدید
