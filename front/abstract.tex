
% -------------------------------------------------------
%  Abstract
% -------------------------------------------------------


\pagestyle{empty}

\شروع{وسط‌چین}
\مهم{چکیده}
\پایان{وسط‌چین}
\بدون‌تورفتگی
نرم‌افزارهایی که کاربر برای استفاده از آن‌ها نیازمند احراز هویت\پانویس{Authentication} است روز به روز در حال افزایش هستند. با توجه به اینکه انتخاب گذرواژه‌ای\پانویس{Password} یکسان برای تمامی این نرم‌افزارها از لحاظ امنیتی کاری نادرست است نیاز به نرم‌افزارهای مدیریت گذرواژه\پانویس{Password Manager} روز به روز بیشتر حس می‌شود. در ابتدا این نرم‌افزارها فقط گذرواژه‌ها را به صورت خام ذخیره می‌کردند، ولی با توجه به نیازهای امنیتی و کاربردی، قابلیت‌هایی مانند رمزنگاری گذرواژه‌ها و ارائه‌ی واسط‌های برنامه‌نویسی\پانویس{API} نیز به آن‌ها اضافه شد.

امروزه یکی از مشکلاتی که نرم‌افزارهای مدیریت گذرواژه را تهدید می‌کند مقاوم نبودن در مقابل Keyloggerهاست. به این معنا که اگر Keyloggerی بر روی سیستم شما نصب باشد و گذرواژه‌ی اصلی (که در واقع کلید دسترسی به بقیه‌ی گذرواژ‌ه‌هاست) را به دست آورد، می‌تواند به تمامی رمز‌های عبور شما دسترسی پیدا کند و این تهدید بسیار بزرگی محسوب می‌شود.
 
در این پایان‌نامه روشی ارائه شده است که طی آن می‌توان نرم‌افزار مدیریت گذرواژه‌ای برای سیستم‌های مبتنی بر گنو/لینوکس\پانویس{GNU/Linux} ساخت که در مقابل keylogger و دیگر خطر‌های احتمالی تا حد امکان مقاوم باشد.
 
\پرش‌بلند
\بدون‌تورفتگی \مهم{کلیدواژه‌ها}: 
نرم‌افزار مدیریت گذرواژه، keylogger، امنیت داده، احراز هویت، گنو/لینوکس
\صفحه‌جدید
